%% start of file `template-zh.tex'.
%% Copyright 2006-2013 Xavier Danaux (xdanaux@gmail.com).
%
% This work may be distributed and/or modified under the
% conditions of the LaTeX Project Public License version 1.3c,
% available at http://www.latex-project.org/lppl/.


\documentclass[11pt,a4paper,roman]{moderncv} % possible options include font size ('10pt', '11pt' and '12pt'), paper size ('a4paper', 'letterpaper', 'a5paper', 'legalpaper' , 'executivepaper' and 'landscape') and font family ('sans' and 'roman')

% moderncv 主題
\moderncvicons{awesome}

\moderncvstyle{banking} % 選項參數是 ‘casual’, ‘classic’, ‘oldstyle’ 和 ’banking’
\moderncvcolor{black} % 選項參數是 ‘blue’ (默認)、‘orange’、‘green’、‘red’、‘purple’ 和 ‘grey’
\nopagenumbers{} % 消除註釋以取消自動頁碼生成功能

% 字符編碼
\usepackage[utf8]{inputenc} % 替換你正在使用的編碼
\usepackage{CJKutf8}

% 調整頁邊距
\usepackage[scale=0.75]{geometry}
%\setlength{\hintscolumnwidth}{3cm} % 如果你希望改變日期欄的寬度

% 個人信息
\name{}{劉昱嫻}
\title{資料科學家} % 可選項、如不需要可刪除本行
\address{莊敬路361號4樓}{110 台北市} % 可選項、如不需要可刪除本行
\phone[mobile]{+1~(234)~567~890} % 可選項、如不需要可刪除本行
\phone[fixed]{+2~(345)~678~901} % 可選項、如不需要可刪除本行
\phone[fax]{+3~(456)~789~012} % 可選項、如不需要可刪除本行
\email{xiaolong@li.com.cn} % 可選項、如不需要可刪除本行
\homepage{www.xialongli.com} % 可選項、如不需要可刪除本行
% \extrainfo{附加信息 (可選項)} % 可選項、如不需要可刪除本行
% \photo[64pt][0.4pt]{picture} % '64pt'是圖片必須壓縮至的高度、'0.4pt'是圖片邊框的寬度(如不需要可調節至0pt)、'picture' 是圖片文件的名字;可選項、如不需要可刪除本行
% \quote{引言(可選項)} % 可選項、如不需要可刪除本行

% 顯示索引號;僅用於在簡歷中使用了引言
%\makeatletter
%\renewcommand*{\bibliographyitemlabel}{\@biblabel{\arabic{enumiv}}}
%\makeatother

% 分類索引
%\usepackage{multibib}
%\newcites{book,misc}{{Books},{Others}}
%------------------------------------------------- ---------------------------------
% 內容
%------------------------------------------------- ---------------------------------
\begin{document}
\begin{CJK*}{UTF8}{bkai} % 詳情參閱CJK文件包
\maketitle

\section{資格}
\cvitem {\faKey \, \emph {git}到OOP語言的掌握能} {}
\begin{itemize}
    \item本機Python和R揚聲器。也是功能強大的Java發言人
    \item具有Docker和版本控制工具的技術團隊成員,可快速投入任何項目
\end{itemize}

\cvitem {\faDatabase \, 具有大數據和機器學習解決方案的經驗} {
\begin{itemize}
    \item 在勞動力市場計劃數據平台(LMPDP)上的大規模(> 200萬條記錄)全棧操作
    \item 對新數據產品的開發和實施以及主要產品的增強充滿熱情
    \item 精通具有Azure-blob和VM的數據倉庫,以整合雲計算和存儲
\end {itemize}
}
\cvitem{\faInfoCircle \, 有效溝通} {}
\begin{itemize}
    \item 制定分析計劃和建議以提供官方統計數據的替代方法
    \item 使用Tableau,Bokeh和Plotly提供了多種交互式數據可視化
\end{itemize}

% \section{畢業論文}
% \cvitem{題目}{\emph{題目}}
% \cvitem{導師}{導師}
% \cvitem{說明}{\small 論文簡介}

\section{計算機技能}
\cvskill{Python, R, Git, Markdown}{5}
\cvskill{Matlab, Tableau, Latex}{3}
\cvskill{Shell scripting, Java, SAS}{2}

\section{教育背景}
\cventry{2018年 -- 2019年}{\textbf{經濟與資料科學 碩士}}{卡爾頓大學}{加拿大渥太華}{}{說明} % 第3到第6編碼可留白
\cventry{2013年 -- 2017年}{數學與經濟 學士}{渥太華大學}{加拿大渥太華}{}{說明}

\section{工作經驗}
\subsection{專業}
\cventry{2019年 -- 現在}{職位}{公司}{加拿大\,渥太華}{}{不超過1--2行的概況說明\newline{}%
工作內容:%
\begin{itemize}%
\item 工作內容 1;
\item 工作內容 2、 含二級內容:
  \begin{itemize}%
  \item 二級內容 (a);
  \item 二級內容 (b)、含三級內容 (不建議使用);
    \begin{itemize}
    \item 三級內容 i;
    \item 三級內容 ii;
    \item 三級內容 iii;
    \end{itemize}
  \item 二級內容 (c);
  \end{itemize}
\item 工作內容 3。
\end{itemize}}
\cventry{年 -- 年}{職位}{公司}{城市}{}{說明行1\newline{}說明行2}
\subsection{其他}
\cventry{2019年}{職位}{公司}{城市}{}{說明}

\section{語言}
\cvskill{英文}{5}
\cvskill{中文}{5}
\cvskill{法文}{1}



% \section{個人興趣}
% \cvitem{愛好 1}{\small 說明}
% \cvitem{愛好 2}{\small 說明}
% \cvitem{愛好 3}{\small 說明}

% \section{其他 1}
% \cvlistitem{項目 1}
% \cvlistitem{項目 2}
% \cvlistitem{項目 3}

% \renewcommand{\listitemsymbol}{-} % 改變列表符號

% \section{其他 2}
% \cvlistdoubleitem{項目 1}{項目 4}
% \cvlistdoubleitem{項目 2}{項目 5\cite{book1}}
% \cvlistdoubleitem{項目 3}{}

% 來自BibTeX文件但不使用multibib包的出版物
%\renewcommand*{\bibliographyitemlabel}{\@biblabel{\arabic{enumiv}}}% BibTeX的數字標籤
\nocite{*}
\bibliographystyle{plain}
\bibliography{publications} % 'publications' 是BibTeX文件的文件名

% 來自BibTeX文件並使用multibib包的出版物
%\section{出版物}
%\nocitebook{book1,book2}
%\bibliographystylebook{plain}
%\bibliographybook{publications} % 'publications' 是BibTeX文件的文件名
%\nocitemisc{misc1,misc2,misc3}
%\bibliographystylemisc{plain}
%\bibliographymisc{publications} % 'publications' 是BibTeX文件的文件名

\clearpage
\end{CJK*}
\end{document}


%% 文件結尾 `template-zh.tex'.